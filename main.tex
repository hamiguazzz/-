\documentclass[a4paper]{ctexart}

\makeatletter
\let\@afterindenttrue\@afterindentfalse
\makeatother
\setlength{\parindent}{0em}

\usepackage{amsmath,amsfonts,amssymb}

\title{数学物理方法\quad 期末考试}
\author{周子正\quad 学号:1810334}
\date{2020 年 6 月}
\begin{document}
\maketitle
\section{第一题}
数学物理方法课程分成两个部分,一是复变函数论,二是数学物理方程。
\setlength{\parindent}{2em}

第一部分首先从复数的引入开始,拓展到复函数。与实分析类似,讨论复变函数的导数和积分,并针对复函数的性质引入柯西积分公式。​在复变函数领域,我们可以通过洛朗展开将泰勒展开进行推广,通过洛朗展开,我们不仅能奇点进行更好的分类,还可以更方便的计算包围奇点区域上的积分,即留数定理,这在某种意义上也是柯西公式的推广。我们利用留数定理不仅能解决复积分,也对实变的积分起了很大帮助。接下来我们将实变函的傅立叶级数推广到到复变函数领域,并从离散的求和变成了连续的积分,得到了傅立叶变换。为了解决奇点上积分无穷大问题和$\exp(ikx)$的变换,引入了广义函数——$\delta$函数,为了解决一部分函数不能变换的问题,我们引入拉普拉斯变换,并分别研究了它们的性质和应用,许多问题因此而变得简单。

第二部分从介绍一些物理情形引入偏微分方程开始,并引入了初始条件和边界条件给出了定解条件,介绍了分类,并针对简单的(半)无界可分离算子情况给出了达朗贝尔公式。此后的讨论主要建立在分离变量的基础上,研究了许多不同方程分离变量的结果。介绍的解PDE一般步骤为利用分离变量法,将偏微分方程转化为带有未定常数和约束条件的常微分方程,即本征值问题,由于讨论到的本征值问题都是施图姆刘维尔型,解具有正交完备性,可以做广义复立叶展开适配边界条件和初始条件,得到解的一般形式。

基于在第一部分讨论的级数和奇点知识,利用常点邻域上的级数解法和正则奇点邻域上的级数解法,求出了勒让德函数和贝塞尔函数,在求解分离变量后得到的本征值方程中有重要应用。

接下来与特殊函数课类似详细研究了勒让德函数和贝塞尔函数(球柱函数)的性质:表示,模长,渐进行为,正交性,广义傅里叶展开(其实是S-L问题的性质),对于贝塞尔函数还有其零点,以及二者一些变形表示。

一句话简单总结为列PDE与定解条件+分离变量+本征值方程+正交完备特殊函数+广义傅里叶展开



\setlength{\parindent}{0em}


​

\section{第二题}

$$
    \begin{aligned}
        z & =\frac{1-i\tan x}{1+i\tan x}=\frac{\cos x-i\sin x}{\cos x+i\sin x}=\frac{-i\left[ \cos \left( \pi /2-x \right) +i\sin \left( \pi /2-x \right) \right]}{\exp \left( ix \right)} \\
          & =\exp \left( i\frac{3\pi}{2} \right) \exp \left( i\frac{\pi}{2} \right) \exp \left( -2ix \right) =\exp \left( -2ix \right)                                                     \\
          & =\cos 2x-i\sin 2x
    \end{aligned}
$$


\section{第三题}

$$
    1+i=\sqrt{2}\exp \left( i\frac{\pi}{4} \right)
$$
$$
    1-i=\sqrt{2}\exp \left( -i\frac{\pi}{4} \right)
$$
$$
    \left( 1+i \right) ^{1000}+\left( 1-i \right) ^{1000}=2^{500}\left[ \exp \left( 250\pi i \right) +\exp \left( -250\pi i \right) \right] =2^{501}
$$


\section{第四题}

$$
    \because |\sqrt{3}+i|=|\sqrt{3}-i|
$$
$$
    \therefore |\sqrt{3}+i|^n=|\sqrt{3}-i|^n
$$

二者模长必然相同,若要求二者相等,只需要辐角满足
$$
    \text{arg}\left( \sqrt{3}+i \right) ^n=2k\pi +\text{arg}\left( \sqrt{3}-i \right) ^n,\quad k\in \mathbb{Z}
$$

$$
    \therefore n\frac{\pi}{6}=-n\frac{\pi}{6}+2k\pi
$$
$$
    \therefore n=6k,\quad k\in \mathbb{Z}
$$

所以只需要$n$是整数且能被$6$整除即可。

\section{第五题}
将$z=x+yi$代入$\omega$展开有:
$$\omega=\frac{x^{2}+y^{2}-1}{x^{2}+(y+1)^{2}}-\frac{2 x}{x^{2}+(y+1)^{2}} i$$

\section{第六题}
$$a=\sum_{k=0}^{\infty} \frac{\cos n z}{n !} $$
$$b=\sum_{k=0}^{\infty} \frac{\sin n z}{n !} $$
考虑做一对替换有:
$$
    c_+\equiv a+bi=\sum_{k=0}^{\infty}{\frac{\cos nz+i\sin nz}{n!}}=\sum_{k=0}^{\infty}{\frac{\exp \left( inz \right)}{n!}}=\exp \left( \exp \left( iz \right) \right)
$$
$$
    c_-\equiv a-bi=\sum_{k=0}^{\infty}{\frac{\cos nz-i\sin nz}{n!}}=\sum_{k=0}^{\infty}{\frac{\exp \left( -inz \right)}{n!}}=\exp \left( \exp \left( -iz \right) \right)
$$
其中
$$
    \exp \left( \exp \left( iz \right) \right) =\exp \left( \cos z+i\sin z \right) =e^{\cos z}\left( \cos\sin z+\sin\sin z \right)
$$
$$
    \exp \left( \exp \left( -iz \right) \right) =\exp \left( \cos z-i\sin z \right) =e^{\cos z}\left( \cos\sin z-\sin\sin z \right)
$$
于是由替换的关系可以得到:
$$
    a=\frac{c_++c_-}{2}=e^{\cos z}\cos\sin z
$$
$$
    b=\frac{c_+-c_-}{2i}=-ie^{\cos z}\sin\sin z
$$

\section{第七题}
求
$$
    f\left( z \right) =\frac{1}{e^z-1}
$$
在$z=0$的洛朗级数,很明显此处不解析,考察$n\in N^*$时,
$$
    \underset{z\rightarrow 0}{\lim}\frac{z^n}{e^z-1}=\left\{ \begin{array}{l}
        \begin{matrix}
            1 & n=1 \\
        \end{matrix} \\
        \begin{matrix}
            0 & n>1 \\
        \end{matrix} \\
    \end{array} \right.
$$
因此这是一阶极点,
考虑求解析函数$a(z)=zf(z)$在$z=0$的泰勒展开,设其为
$$
    a\left( z \right) =\sum_{n=0}^{+\infty}{a_nz^n}
$$
这不容易计算,考虑其倒数$b(z)=1/a$,可以展开为
$$
    b\left( z \right) =\sum_{n=0}^{+\infty}{\frac{z^n}{\left( n+1 \right) !}}
$$
又因为恒等关系$a(z)b(z)=1$
$$
    \left( \sum_{n=0}^{+\infty}{a_nz^n} \right) \cdot \left( \sum_{n=0}^{+\infty}{\frac{z^n}{\left( n+1 \right) !}} \right) =1
$$
比较$z$的各阶系数可以得到$a_0=1,a_{1}=-\frac{1}{2}$...

在$n\geq 1$时,其满足线性齐次方程组方程为:
$$
    \sum_{k=0}^n{\frac{a_k}{\left( n-k+1 \right) !}}=0
$$
理论上可以解出任意的$a_n$

我试图找到通项公式但是失败了,查阅资料发现这个数列与伯努利数相关,关系为$a_n=B_n/n!$其被定义为
$$\frac{z}{e^{z}-1}=\sum_{n=0}^{\infty} B_{n} \frac{z^{n}}{n !}$$
递推关系为
$$B_{m}=[m=0]-\sum_{k=0}^{m-1}\left(\begin{array}{c}
            m \\
            k
        \end{array}\right) \frac{B_{k}}{m-k+1}$$
$B_0=1$,其中 $[m=0]$ 表示当 $m=0$ 时,取1,其余取0。

综上可得,
$$
    f\left( z \right) =\sum_{n=-1}^{+\infty}{\frac{z^n}{\left( n+1 \right) !}B_{n+1}}
$$

\section{第八题}
由于所有极点都在围道内部,直接考察无穷远的留数,根据引理有:
$$
    \int_{|z|=200}{f\left( z \right) dz}=-2\pi i\underset{z\rightarrow \infty}{\text{Re}s}f\left( z \right) =2\pi i\underset{t\rightarrow 0}{\text{Re}s}\left[ f\left( \frac{1}{t} \right) \frac{1}{t^2} \right]
$$
其中
$$
    f\left( \frac{1}{t} \right) \frac{1}{t^2}=\frac{1}{t^2}\prod_{k=1}^{100}{\frac{1}{1-kt}}
$$
$$
    \because \underset{t\rightarrow 0}{\lim}t^2\frac{1}{t^2}\prod_{k=1}^{100}{\frac{1}{1-kt}}=1
$$
$$
    \therefore \text{二阶极点}
$$
所以可以计算出留数的值
$$
    \begin{aligned}
        \underset{t\rightarrow 0}{\text{Re}s}\left[ f\left( \frac{1}{t} \right) \frac{1}{t^2} \right]
         & =\underset{t\rightarrow 0}{\lim}\frac{d}{dt}\prod_{k=1}^{100}{\frac{1}{1-kt}}                      \\
         & =\underset{t\rightarrow 0}{\lim}\sum_{n=1}^{100}{\ \prod_{n=1 \land k\ne n}^{100}{\frac{n}{1-kt}}} \\
         & =\sum_{n=1}^{100}{n}=5050
    \end{aligned}
$$
$$
    \therefore \int_{|z|=200}{f\left( z \right) dz}=10100\pi i
$$

\section{第九题}
\section{第十题}
直接考察积分,试图利用柯西公式
$$
    \begin{aligned}
        D_F\left( x-y \right) & =\int_{C_F}{\frac{d^4p}{\left( 2\pi \right) ^4}\frac{ie^{-ip\cdot \left( x-y \right)}}{p^2-m^2}}                                                                                                  \\
                              & =\int_{C_F}{\frac{d^3\vec{p}dp_0}{\left( 2\pi \right) ^4}\frac{ie^{i\vec{p}\cdot \left( \vec{x}-\vec{y} \right)}e^{-ip_0\left( x_0-y_0 \right)}}{p_{0}^{2}-\left( \vec{p}^2+m^2 \right)}}         \\
                              & =\int{\frac{d^3\vec{p}\left[ e^{i\vec{p}\cdot \left( \vec{x}-\vec{y} \right)} \right]}{\left( 2\pi \right) ^4}}\int_{C_F}{\frac{ie^{-ip_0\left( x_0-y_0 \right)}dp_0}{p_{0}^{2}-E_{\vec{p}}^{2}}} \\
    \end{aligned}
$$
考察其中的
$$
    \int_{C_F}{\frac{ie^{-ip_0\left( x_0-y_0 \right)}dp_0}{p_{0}^{2}-E_{\vec{p}}^{2}}}=\int_{C_F}{\frac{ie^{-ip_0\left( x_0-y_0 \right)}dp_0}{\left( p_0-E_{\vec{p}} \right) \left( p_0+E_{\vec{p}} \right)}}
$$
注意到在$C_F$的围道内只有一个一阶极点为
$$
    \underset{p_0\rightarrow E_{\vec{p}}}{\text{Re}s}\frac{ie^{-ip_0\left( x_0-y_0 \right)}}{\left( p_0-E_{\vec{p}} \right) \left( p_0+E_{\vec{p}} \right)}=\frac{ie^{-ip_0\left( x_0-y_0 \right)}}{2E_{\vec{p}}}
$$
考察其在无穷远的性质有
$$
    \underset{|p_0|\rightarrow \infty}{\lim}\left| \frac{ip_0e^{-ip_0\left( x_0-y_0 \right)}}{p_{0}^{2}-E_{\vec{p}}^{2}} \right|=0
$$
实际上是一致收敛的\\
注意到$C_F$的围道方向为顺时针,因此有
$$
    \int_{C_F}{\frac{ie^{-ip_0\left( x_0-y_0 \right)}dp_0}{p_{0}^{2}-E_{\vec{p}}^{2}}}=2\pi \frac{e^{-ip_0\left( x_0-y_0 \right)}}{2E_{\vec{p}}}
$$
代入 Feynman 传播子表示可得到:
$$
    D_F\left( x-y \right) =\int_{C_F}{\frac{d^4p}{\left( 2\pi \right) ^4}\frac{ie^{-ip\cdot \left( x-y \right)}}{p^2-m^2}}=\left. \int{\frac{d^3\vec{p}}{\left( 2\pi \right) ^32E_{\vec{p}}}e^{-ip\cdot \left( x-y \right)}} \right|_{p_0=E_{\vec{p}}}
$$
\section{第十一题}
$$
    S=\int{d}^2x\left\{ -\frac{1}{2}\partial _{\mu}\phi \left( x \right) \partial ^{\mu}\phi \left( x \right) +\frac{1}{2}\lambda _{\mu v}\left( x \right) \left[ \partial ^{\mu}\phi \left( x \right) -\epsilon ^{\mu \sigma}\partial _{\sigma}\phi \left( x \right) \right] \left[ \partial ^v\phi \left( x \right) -\epsilon ^{v\rho}\partial _{\rho}\phi \left( x \right) \right] \right\}
$$
先分别计算其中的积分
$$
    S_1=\int{d}^2x\left[ \partial _{\mu}\phi \left( x \right) \partial ^{\mu}\phi \left( x \right) \right]
$$
$$
    S_2=\int{d}^2x\left\{ \lambda _{\mu v}\left( x \right) \left[ \partial ^{\mu}\phi \left( x \right) -\epsilon ^{\mu \sigma}\partial _{\sigma}\phi \left( x \right) \right] \left[ \partial ^v\phi \left( x \right) -\epsilon ^{v\rho}\partial _{\rho}\phi \left( x \right) \right] \right\}
$$
$$
    \because \phi \left( x \right) =\mathcal{F}\left( \phi \left( k \right) \right) ,\lambda _{\mu v}\left( x \right) =\mathcal{F}\left( \lambda _{\mu v}\left( k \right) \right)
$$
换元,令
$$
    g_{\mu}=-ik_{\mu}\phi \left( k \right) ,g^{\mu}=-ik^{\mu}\phi \left( k \right)
$$
导数定理
$$
    \because \mathcal{F}\left( -ik_{\mu}\phi \left( k \right) \right) =\partial _{\mu}\phi \left( x \right) ,\mathcal{F}\left( -ik^{\mu}\phi \left( k \right) \right) =\partial ^{\mu}\phi \left( x \right)
$$
卷积定理
$$
    S_1=\int{d}^2x\mathcal{F}\left( g_{\mu} \right) \mathcal{F}\left( g^{\mu} \right) =\frac{1}{2\pi}\int{d}^2x\mathcal{F}\left( g_{\mu}*g^{\mu} \right)
$$
$$
    \therefore S_1=\left. \left( g_{\mu}*g^{\mu} \right) \right|_{k=0}=\int{d}^2\xi g_{\mu}\left( \xi \right) g^{\mu}\left( 0-\xi \right) =\int{d}^2k\left( -ik_{\mu} \right) \phi \left( k \right) \left( -ik^{\mu} \right) \phi \left( k \right)
$$
同理我们设
$$
    \eta ^{\mu}=ik^{\mu}\phi \left( k \right) -\epsilon ^{\mu \sigma}ik_{\sigma}\phi \left( k \right) ,\eta ^{\nu}=ik^{\nu}\phi \left( k \right) -\epsilon ^{v\rho}ik_{\rho}\phi \left( k \right)
$$
有
$$
    S_2=\int{d}^2x\mathcal{F}\left( \lambda _{\mu v}\left( k \right) \right) \mathcal{F}\left( \eta ^{\mu} \right) \mathcal{F}\left( \eta ^{\nu} \right) =\frac{1}{4\pi ^2}\int{d}^2x\mathcal{F}\left( \lambda _{\mu v}\left( k \right) *\eta ^{\mu}*\eta ^{\nu} \right)
$$
$$
    S_2=\frac{1}{2\pi}\left. \lambda _{\mu v}\left( k \right) *\eta ^{\mu}*\eta ^{\nu} \right|_{k=0}=\frac{1}{2\pi}\int{d}^2\xi \int{d}^2\zeta \lambda _{\mu v}\left( \zeta +\xi \right) \eta ^{\mu}\left( 0-\xi \right) \eta ^{\nu}\left( 0-\zeta \right)
$$
即为
$$
    S_2=\frac{1}{2\pi}\int{d}^2k\int{d}^2k'\lambda _{\mu v}\left( -k-k' \right) \left[ ik^{\mu}\phi \left( k \right) -\epsilon ^{\mu \sigma}ik_{\sigma}\phi \left( k \right) \right] \left[ ik^{\nu}\phi \left( k' \right) -\epsilon ^{v\rho}ik_{\rho}\phi \left( k' \right) \right]
$$

现在我们终于可以代回到原表达式
$$
    \begin{aligned}
        S_{m}
        = & -\frac{1}{2} \int d^{2} k\left(-i k_{\mu}\right)\left(i k^{\mu}\right) \phi(k) \phi(-k)                                                                                                                                                                      \\
          & +\frac{1}{4 \pi} \int d^{2} k d^{2} k^{\prime} \lambda_{\mu \nu}\left(-k-k^{\prime}\right)\left(i k^{\mu}-\epsilon^{\mu \sigma} i k_{\sigma}\right)\left(i k^{\prime \nu}-\epsilon^{\nu \rho} i k_{\rho}^{\prime}\right) \phi(k) \phi\left(k^{\prime}\right)
    \end{aligned}
$$

这就是结果

\section{第十二题}
\subsection{}
$$
    g_1(t)=\left\{\begin{array}{cc}
        2 t / T    & 0 \leq t<T / 2      \\
        2(1-t / T) & T / 2 \leq t \leq T
    \end{array}\right.
$$
根据周期函数
$$
    \mathcal{L}\left( g\left( t \right) \right) =
    \frac{1}{1-e^{-p T}} \int_{0}^{T} e^{-p t} g(t) d t
$$
计算得
$$
    \int_0^{T/2}{\exp \left( -pt \right) \frac{2t}{T}dt=\frac{2}{T}}\left[ \frac{1}{p^2}-e^{-Tp/2}\left( \frac{T}{2p}+\frac{1}{p^2} \right) \right]
$$
$$
    -\int_{T/2}^T{\exp \left( -pt \right) \frac{2t}{T}dt=\frac{2}{T}}e^{-Tp/2}\left[ \left( 1-e^{-Tp/2} \right) \left( \frac{T}{2p}+\frac{1}{p^2} \right) \right]
$$
$$
    \int_{T/2}^T{2\exp \left( -pt \right) dt=\frac{2}{p}}\left( e^{-Tp/2}-e^{-Tp} \right)
$$
代入化简得
$$
    \mathcal{L}\left( g_1\left( t \right) \right) =\frac{2}{Tp^2}\tan\text{h}\left( \frac{Tp}{4} \right)
$$
\subsection{}
$$
    g_2\left( t \right) =\frac{2}{T}\left[ \text{tH}\left( \text{t} \right) +2\sum_{n=1}^{\infty}{\left( -1 \right)}^n\left( t-\frac{1}{2}nT \right) H\left( t-\frac{1}{2}nT \right) \right]
$$
$$
    \mathcal{L}\left( g_2\left( t \right) \right) =\frac{2}{Tp^2}\left[ 1+2\sum_{n=1}^{\infty}{\left( -1 \right)}^n\exp \left( -\frac{1}{2}nT \right) \right]
$$
如果$g_1(t)=g_2(t)$,比照系数令$x=Tp/4$换元有
$$
    \tanh x=1+2 \sum_{n=0}^{\infty}(-1)^{n} e^{-2 n x}
$$
这正是我们想证明的。因此只需证明$g_1(t)=g_2(t)$.

下面证明$g_1(t)=g_2(t)$

我们首先考察区间$[0,T/2)$,对于$t\in [0,T/2]$由于全部的$H(t-1/2nT)=0$因此二者显然相等。

然后考察区间$[T/2,T)$,对于$t\in [0,T/2]$求和只保留到$n=1$,有
$$
    g_2=\frac{2}{T}\left[ t-2\left( t-\frac{1}{2}T \right) \right] =2-\frac{2t}{T}=g_1
$$

综上二者在$t\in [0,T)$上相等。

一般地,对于$n>0$情形,仅当$t\geq 1/2nT$时,$H(t-1/2nT)=1$,其余情况为0。如果我们只考察$t$非负半轴情况,求和可以写成:
$$
    g_2\left( t \right) =\frac{2}{T}\left[ t +2\sum_{n=1}^{\left[ \frac{2t}{T} \right]}{\left( -1 \right)}^n\left( t-\frac{1}{2}nT \right) \right]
$$
其中$\left[ \frac{2t}{T} \right]$表示括号内值向下取整。

接下来证明$g_2(t)$也以$T$为周期,既$g_2(t)=g_2(t+T)$
$$
    \begin{aligned}
        g_2\left( t+T \right)
         & =\frac{2}{T}\left[ \text{t}+T+2\sum_{n=1}^{\left[ \frac{2t}{T}+2 \right]}{\left( -1 \right)}^n\left( t-\frac{1}{2}nT+T \right) \right]                         \\
         & =\frac{2}{T}\left[ \text{t}+T-\left( 2t+T \right) +2t+2\sum_{n=3}^{\left[ \frac{2t}{T}+2 \right]}{\left( -1 \right)}^n\left( t-\frac{1}{2}nT+T \right) \right] \\
         & =\frac{2}{T}\left[ t+2\sum_{n=1}^{\left[ \frac{2t}{T} \right]}{\left( -1 \right)}^n\left( t-\frac{1}{2}nT \right) \right]                                      \\
         & =g_2\left( t \right)
    \end{aligned}
$$

由于在$[0,T)$区间相等并都具有以$T$为周期的周期性,因此$g_1(t)=g_2(t)$,这正是我们需要的。
于是
$$
    \tanh x=1+2 \sum_{n=0}^{\infty}(-1)^{n} e^{-2 n x}
$$
得证


\section{第十三题}
\section{第十四题}
\end{document}