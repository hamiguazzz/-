\documentclass[a4paper]{ctexart}

\makeatletter
\let\@afterindenttrue\@afterindentfalse
\makeatother
\setlength{\parindent}{0em}

\usepackage{amsthm}
\newtheorem{yinli}[subsection]{引理}

\usepackage{amsmath,amsfonts,amssymb}

\title{数学物理方法\quad 期末考试}
\author{周子正\quad 学号:1810334}
\date{2020 年 6 月}
\begin{document}
\maketitle
\section{第一题}
数学物理方法课程分成两个部分,一是复变函数论,二是数学物理方程。
\setlength{\parindent}{2em}

第一部分首先从复数的引入开始,拓展到复函数。与实分析类似,讨论复变函数的导数和积分,并针对复函数的性质引入柯西积分公式。​在复变函数领域,我们可以通过洛朗展开将泰勒展开进行推广,通过洛朗展开,我们不仅能奇点进行更好的分类,还可以更方便的计算包围奇点区域上的积分,即留数定理,这在某种意义上也是柯西公式的推广。我们利用留数定理不仅能解决复积分,也对实变的积分起了很大帮助。接下来我们将实变函的傅立叶级数推广到到复变函数领域,并从离散的求和变成了连续的积分,得到了傅立叶变换。为了解决奇点上积分无穷大问题和$\exp(ikx)$的变换,引入了广义函数——$\delta$函数,为了解决一部分函数不能变换的问题,我们引入拉普拉斯变换,并分别研究了它们的性质和应用,许多问题因此而变得简单。

第二部分从介绍一些物理情形引入偏微分方程开始,并引入了初始条件和边界条件给出了定解条件,介绍了分类,并针对简单的(半)无界可分离算子情况给出了达朗贝尔公式。此后的讨论主要建立在分离变量的基础上,研究了许多不同方程分离变量的结果。介绍的解PDE一般步骤为利用分离变量法,将偏微分方程转化为带有未定常数和约束条件的常微分方程,即本征值问题,由于讨论到的本征值问题都是施图姆刘维尔型,解具有正交完备性,可以做广义复立叶展开适配边界条件和初始条件,得到解的一般形式。

基于在第一部分讨论的级数和奇点知识,利用常点邻域上的级数解法和正则奇点邻域上的级数解法,求出了勒让德函数和贝塞尔函数,在求解分离变量后得到的本征值方程中有重要应用。

接下来与特殊函数课类似详细研究了勒让德函数和贝塞尔函数(球柱函数)的性质:表示,模长,渐进行为,正交性,广义傅里叶展开(其实是S-L问题的性质),对于贝塞尔函数还有其零点,以及二者一些变形表示。

一句话简单总结为列PDE与定解条件+分离变量+本征值方程+正交完备特殊函数+广义傅里叶展开



\setlength{\parindent}{0em}


​

\section{第二题}

$$
    \begin{aligned}
        z & =\frac{1-i\tan x}{1+i\tan x}=\frac{\cos x-i\sin x}{\cos x+i\sin x}=\frac{-i\left[ \cos \left( \pi /2-x \right) +i\sin \left( \pi /2-x \right) \right]}{\exp \left( ix \right)} \\
          & =\exp \left( i\frac{3\pi}{2} \right) \exp \left( i\frac{\pi}{2} \right) \exp \left( -2ix \right) =\exp \left( -2ix \right)                                                     \\
          & =\cos 2x-i\sin 2x
    \end{aligned}
$$


\section{第三题}

$$
    1+i=\sqrt{2}\exp \left( i\frac{\pi}{4} \right)
$$
$$
    1-i=\sqrt{2}\exp \left( -i\frac{\pi}{4} \right)
$$
$$
    \left( 1+i \right) ^{1000}+\left( 1-i \right) ^{1000}=2^{500}\left[ \exp \left( 250\pi i \right) +\exp \left( -250\pi i \right) \right] =2^{501}
$$


\section{第四题}

$$
    \because |\sqrt{3}+i|=|\sqrt{3}-i|
$$
$$
    \therefore |\sqrt{3}+i|^n=|\sqrt{3}-i|^n
$$

二者模长必然相同,若要求二者相等,只需要辐角满足
$$
    \text{arg}\left( \sqrt{3}+i \right) ^n=2k\pi +\text{arg}\left( \sqrt{3}-i \right) ^n,\quad k\in \mathbb{Z}
$$

$$
    \therefore n\frac{\pi}{6}=-n\frac{\pi}{6}+2k\pi
$$
$$
    \therefore n=6k,\quad k\in \mathbb{Z}
$$

所以只需要$n$是整数且能被$6$整除即可。

\section{第五题}
平面上圆的一般方程可以写成
$$
\alpha _1\left( x^2+y^2 \right) +2\alpha _2x+2\alpha _3y+\alpha _4=0
$$
其中$\alpha_{1,2,3,4}\in\mathbb{R}$,若$\alpha_1=0$,其退化为直线

若代入
$$
x=\frac{z+\bar{z}}{2},y=\frac{z-\bar{z}}{2i}
$$
设$\beta =\alpha _2+i\alpha _3$有复数表示
$$
\alpha _1z\bar{z}+\bar{\beta}z+\beta \bar{z}+\alpha _4=0
$$
对于此题,
$$
w=\frac{z-i}{z+i}\Rightarrow z=-\frac{\left( w+1 \right) i}{w-1},\bar{z}=\frac{\left( \bar{w}+1 \right) i}{\bar{w}-1}
$$
代入化简有
$$
\gamma _1w\bar{w}+\bar{\beta}_2w+\beta _2\bar{w}+\gamma _4=0
$$
其中
$$
\left( \begin{array}{c}
	\gamma _1\\
	\gamma _2\\
	\gamma _3\\
	\gamma _4\\
\end{array} \right) =\left( \begin{array}{c}
	\alpha _1-2\alpha _3+\alpha _4\\
	\alpha _1-\alpha _4\\
	-2\alpha _2\\
	\alpha _1+2\alpha _3+\alpha _4\\
\end{array} \right) \qquad \beta _2=\gamma _2+i\gamma _3
$$

因此广义上(直线是特殊的圆)看,$w$将圆变成圆。特别的,对于此题的情况有:

$$
\left( \begin{matrix}
	\alpha _1&		\alpha _2&		\alpha _3&		\alpha _4\\
\end{matrix} \right) =\left( \begin{matrix}
	0&		1/2&		0&		-a\\
\end{matrix} \right) \rightarrow \left( \begin{matrix}
	\gamma _1&		\gamma _2&		\gamma _3&		\gamma _4\\
\end{matrix} \right) =\left( \begin{matrix}
	-a&		a&		-1&		-a\\
\end{matrix} \right) 
$$
$$
\left( \begin{matrix}
	\alpha _1&		\alpha _2&		\alpha _3&		\alpha _4\\
\end{matrix} \right) =\left( \begin{matrix}
	0&		0&		1/2&		-b\\
\end{matrix} \right) \rightarrow \left( \begin{matrix}
	\gamma _1&		\gamma _2&		\gamma _3&		\gamma _4\\
\end{matrix} \right) =\left( \begin{matrix}
	-1-b&		b&		0&		1-b\\
\end{matrix} \right) 
$$
显然当且仅当$-a=0$,$-1-b=0$即$a=0$,$b=1$时,退化为直线。
\section{第六题}
$$a=\sum_{k=0}^{\infty} \frac{\cos n z}{n !} $$
$$b=\sum_{k=0}^{\infty} \frac{\sin n z}{n !} $$
考虑做一对替换有:
$$
    c_+\equiv a+bi=\sum_{k=0}^{\infty}{\frac{\cos nz+i\sin nz}{n!}}=\sum_{k=0}^{\infty}{\frac{\exp \left( inz \right)}{n!}}=\exp \left( \exp \left( iz \right) \right)
$$
$$
    c_-\equiv a-bi=\sum_{k=0}^{\infty}{\frac{\cos nz-i\sin nz}{n!}}=\sum_{k=0}^{\infty}{\frac{\exp \left( -inz \right)}{n!}}=\exp \left( \exp \left( -iz \right) \right)
$$
其中
$$
    \exp \left( \exp \left( iz \right) \right) =\exp \left( \cos z+i\sin z \right) =e^{\cos z}\left( \cos\sin z+\sin\sin z \right)
$$
$$
    \exp \left( \exp \left( -iz \right) \right) =\exp \left( \cos z-i\sin z \right) =e^{\cos z}\left( \cos\sin z-\sin\sin z \right)
$$
于是由替换的关系可以得到:
$$
    a=\frac{c_++c_-}{2}=e^{\cos z}\cos\sin z
$$
$$
    b=\frac{c_+-c_-}{2i}=-ie^{\cos z}\sin\sin z
$$

\section{第七题}
求
$$
    f\left( z \right) =\frac{1}{e^z-1}
$$
在$z=0$的洛朗级数,很明显此处不解析,考察$n\in N^*$时,
$$
    \underset{z\rightarrow 0}{\lim}\frac{z^n}{e^z-1}=\left\{ \begin{array}{l}
        \begin{matrix}
            1 & n=1 \\
        \end{matrix} \\
        \begin{matrix}
            0 & n>1 \\
        \end{matrix} \\
    \end{array} \right.
$$
因此这是一阶极点,
考虑求解析函数$a(z)=zf(z)$在$z=0$的泰勒展开,设其为
$$
    a\left( z \right) =\sum_{n=0}^{+\infty}{a_nz^n}
$$
这不容易计算,考虑其倒数$b(z)=1/a$,可以展开为
$$
    b\left( z \right) =\sum_{n=0}^{+\infty}{\frac{z^n}{\left( n+1 \right) !}}
$$
又因为恒等关系$a(z)b(z)=1$
$$
    \left( \sum_{n=0}^{+\infty}{a_nz^n} \right) \cdot \left( \sum_{n=0}^{+\infty}{\frac{z^n}{\left( n+1 \right) !}} \right) =1
$$
比较$z$的各阶系数可以得到$a_0=1,a_{1}=-\frac{1}{2}$...

在$n\geq 1$时,其满足线性齐次方程组方程为:
$$
    \sum_{k=0}^n{\frac{a_k}{\left( n-k+1 \right) !}}=0
$$
理论上可以解出任意的$a_n$

我试图找到通项公式但是失败了,查阅资料发现这个数列与伯努利数相关,关系为$a_n=B_n/n!$其被定义为
$$\frac{z}{e^{z}-1}=\sum_{n=0}^{\infty} B_{n} \frac{z^{n}}{n !}$$
递推关系为
$$B_{m}=[m=0]-\sum_{k=0}^{m-1}\left(\begin{array}{c}
            m \\
            k
        \end{array}\right) \frac{B_{k}}{m-k+1}$$
$B_0=1$,其中 $[m=0]$ 表示当 $m=0$ 时,取1,其余取0。

综上可得,
$$
    f\left( z \right) =\sum_{n=-1}^{+\infty}{\frac{z^n}{\left( n+1 \right) !}B_{n+1}}
$$
其前几项为
$$
\frac{1}{e^x-1}=\frac{1}{z}-\frac{1}{2}+\frac{z}{12}-\frac{z^3}{720}+\cdots 
$$

\section{第八题}
由于所有极点都在围道内部,直接考察无穷远的留数,根据引理有:
$$
    \int_{|z|=200}{f\left( z \right) dz}=-2\pi i\underset{z\rightarrow \infty}{\text{Re}s}f\left( z \right) =2\pi i\underset{t\rightarrow 0}{\text{Re}s}\left[ f\left( \frac{1}{t} \right) \frac{1}{t^2} \right]
$$
其中
$$
    f\left( \frac{1}{t} \right) \frac{1}{t^2}=\frac{1}{t^2}\prod_{k=1}^{100}{\frac{1}{1-kt}}
$$
$$
    \because \underset{t\rightarrow 0}{\lim}t^2\frac{1}{t^2}\prod_{k=1}^{100}{\frac{1}{1-kt}}=1
$$
$$
    \therefore \text{二阶极点}
$$
所以可以计算出留数的值
$$
    \begin{aligned}
        \underset{t\rightarrow 0}{\text{Re}s}\left[ f\left( \frac{1}{t} \right) \frac{1}{t^2} \right]
         & =\underset{t\rightarrow 0}{\lim}\frac{d}{dt}\prod_{k=1}^{100}{\frac{1}{1-kt}}                      \\
         & =\underset{t\rightarrow 0}{\lim}\sum_{n=1}^{100}{\ \prod_{n=1 \land k\ne n}^{100}{\frac{n}{1-kt}}} \\
         & =\sum_{n=1}^{100}{n}=5050
    \end{aligned}
$$
$$
    \therefore \int_{|z|=200}{f\left( z \right) dz}=10100\pi i
$$

\section{第九题}
考虑配对,设
$$
    \begin{aligned}
        I_1      & =\int_0^{2\pi}{e}^{\cos \theta}\cos \left( n\theta -\sin \theta \right) d\theta                    \\
        I_2      & =\int_0^{2\pi}{e}^{\cos \theta}\sin \left( n\theta -\sin \theta \right) d\theta                    \\
        I_1+iI_2 & =\int_0^{2\pi}{\exp \left( \cos \theta -i\sin \theta \right) \exp \left( in\theta \right)}d\theta  \\
        I_1-iI_2 & =\int_0^{2\pi}{\exp \left( \cos \theta +i\sin \theta \right) \exp \left( -in\theta \right)}d\theta
    \end{aligned}
$$
对于$I_1+iI_2$设$z=e^{i\theta}$,有
$$
    I_1+iI_2=\int_{\left| z \right|=1}{e^{\frac{1}{z}}z^n\frac{dz}{zi}=2\pi \underset{z\rightarrow 0}{\text{Re}s}e^{\frac{1}{z}}z^{n-1}}
$$
包裹的奇点只在$z=0$,我们计算其在$z=0$洛朗展开为
$$
    e^{\frac{1}{z}}z^{n-1}=\sum_{k=-\infty}^0{\frac{e^{n+k-1}}{\left( -k \right) !}}
$$
为求其留数,取$z^{-1}$项要求$k=-n$所以有
$$
    \underset{z\rightarrow 0}{\text{Re}s}e^{\frac{1}{z}}z^{n-1}=\frac{1}{n!}
$$
对于$I_1-iI_2$设$z=e^{-i\theta}$,注意积分方向相反需要取出一个负号,化简有
$$
    I_1-iI_2=\int_{\left| z \right|=1}{e^{\frac{1}{z}}z^n\frac{dz}{zi}=}I_1+iI_2=\frac{2\pi}{n!}
$$
于是要求的量即为
$$
    I_1=\frac{2\pi}{n!}
$$

\section{第十题}
直接考察积分,试图利用柯西公式
$$
    \begin{aligned}
        D_F\left( x-y \right) & =\int_{C_F}{\frac{d^4p}{\left( 2\pi \right) ^4}\frac{ie^{-ip\cdot \left( x-y \right)}}{p^2-m^2}}                                                                                                  \\
                              & =\int_{C_F}{\frac{d^3\vec{p}dp_0}{\left( 2\pi \right) ^4}\frac{ie^{i\vec{p}\cdot \left( \vec{x}-\vec{y} \right)}e^{-ip_0\left( x_0-y_0 \right)}}{p_{0}^{2}-\left( \vec{p}^2+m^2 \right)}}         \\
                              & =\int{\frac{d^3\vec{p}\left[ e^{i\vec{p}\cdot \left( \vec{x}-\vec{y} \right)} \right]}{\left( 2\pi \right) ^4}}\int_{C_F}{\frac{ie^{-ip_0\left( x_0-y_0 \right)}dp_0}{p_{0}^{2}-E_{\vec{p}}^{2}}} \\
    \end{aligned}
$$
考察其中的
$$
    \int_{C_F}{\frac{ie^{-ip_0\left( x_0-y_0 \right)}dp_0}{p_{0}^{2}-E_{\vec{p}}^{2}}}=\int_{C_F}{\frac{ie^{-ip_0\left( x_0-y_0 \right)}dp_0}{\left( p_0-E_{\vec{p}} \right) \left( p_0+E_{\vec{p}} \right)}}
$$
注意到在$C_F$的围道内只有一个一阶极点为
$$
    \underset{p_0\rightarrow E_{\vec{p}}}{\text{Re}s}\frac{ie^{-ip_0\left( x_0-y_0 \right)}}{\left( p_0-E_{\vec{p}} \right) \left( p_0+E_{\vec{p}} \right)}=\frac{ie^{-ip_0\left( x_0-y_0 \right)}}{2E_{\vec{p}}}
$$
考察其在无穷远的性质有
$$
    \underset{|p_0|\rightarrow \infty}{\lim}\left| \frac{ip_0e^{-ip_0\left( x_0-y_0 \right)}}{p_{0}^{2}-E_{\vec{p}}^{2}} \right|=0
$$
实际上是一致收敛的\\
注意到$C_F$的围道方向为顺时针,因此有
$$
    \int_{C_F}{\frac{ie^{-ip_0\left( x_0-y_0 \right)}dp_0}{p_{0}^{2}-E_{\vec{p}}^{2}}}=2\pi \frac{e^{-ip_0\left( x_0-y_0 \right)}}{2E_{\vec{p}}}
$$
代入 Feynman 传播子表示可得到:
$$
    D_F\left( x-y \right) =\int_{C_F}{\frac{d^4p}{\left( 2\pi \right) ^4}\frac{ie^{-ip\cdot \left( x-y \right)}}{p^2-m^2}}=\left. \int{\frac{d^3\vec{p}}{\left( 2\pi \right) ^32E_{\vec{p}}}e^{-ip\cdot \left( x-y \right)}} \right|_{p_0=E_{\vec{p}}}
$$
\section{第十一题}
以下都建立在$D=2$上
\begin{yinli}{导数定理}\label{11dao}
    $$
        f\left( x \right) =\mathcal{F}\left( g\left( k \right) \right) \Rightarrow \partial _{\mu}f\left( x \right) =\mathcal{F}\left( -ik_{\mu}g\left( k \right) \right)
    $$
\end{yinli}
\begin{proof}
    $$
        \mathcal{F}^{-1}\left( \partial _{\mu}f\left( x \right) \right) =\frac{1}{2\pi}\int{d^2x\exp \left( ik\cdot x \right)}\partial _{\mu}f\left( x \right)
    $$
    $$
        =-\frac{1}{2\pi}\int{d^2x\left( ik_{\mu} \right) \exp \left( ik\cdot x \right)}f\left( x \right) =-ik_{\mu}g\left( k \right)
    $$
    $$
        \therefore \partial _{\mu}f\left( x \right) =\mathcal{F}\left( -ik_{\mu}g\left( k \right) \right)
    $$
    同理,
    $$
        \partial ^{\mu}f\left( x \right) =\mathcal{F}\left( -ik^{\mu}g\left( k \right) \right)
    $$

\end{proof}
\begin{yinli}{卷积定理}\label{11juan}

    如果定义
    $$
        f*g\left( x \right) \equiv \int{d^2\xi f\left( \xi \right)}g\left( x-\xi \right)
    $$
    则有
    $$
        \mathcal{F}\left( f*g\left( x \right) \right) =2\pi \mathcal{F}\left( f\left( x \right) \right) \mathcal{F}\left( g\left( x \right) \right)
    $$
\end{yinli}
\begin{proof}
    $$
        \begin{aligned}
            \mathcal{F}\left( f*g\left( x \right) \right)
             & =\frac{1}{2\pi}\int{d^2xe^{-ik\cdot x}}\int{d^2\xi f\left( \xi \right)}g\left( x-\xi \right)                \\
             & =\int{d^2\xi}f\left( \xi \right) \left[ \frac{1}{2\pi}\int{d^2x}e^{-ik\cdot x}g\left( x-\xi \right) \right] \\
             & =\mathcal{F}\left( g\left( x \right) \right) \int{d^2\xi}f\left( \xi \right) e^{-ik\cdot \xi}               \\
             & =2\pi \mathcal{F}\left( f\left( x \right) \right) \mathcal{F}\left( g\left( x \right) \right)
        \end{aligned}
    $$
    推论:对于三个函数的情形,只需要依次计算,即为
    $$
        \mathcal{F}\left( f*g*h\left( x \right) \right) =4\pi ^2\mathcal{F}\left( f\left( x \right) \right) \mathcal{F}\left( g\left( x \right) \right) \mathcal{F}\left( h\left( x \right) \right)
    $$


\end{proof}
\begin{yinli}\label{11delta}
    设对于$f(x)$有$f(x)=\mathcal{F}(g(k))$则,
    $$
        \int{d^2x\mathcal{F}\left( g\left( k \right) \right) }=\left( 2\pi \right) ^2g\left( 0 \right)
    $$
\end{yinli}
\begin{proof}
    $$
        \begin{aligned}
            \int{d^2x\mathcal{F}\left( g\left( k \right) \right)}
             & =\int{d^2x}\int{d^2k\exp \left( -ik\cdot x \right) g\left( k \right)}                                                \\
             & =\int{d^2kg\left( k \right)}\int{d^2x\exp \left( -ik\cdot x \right)}                                                 \\
             & =\left( 2\pi \right) ^2\int{d^2kg\left( k \right) \delta \left( -k \right) =}\left( 2\pi \right) ^2g\left( 0 \right)
        \end{aligned}
    $$

\end{proof}
我们考虑原始表达式
$$
    S=\int{d}^2x\left\{ -\frac{1}{2}\partial _{\mu}\phi \left( x \right) \partial ^{\mu}\phi \left( x \right) +\frac{1}{2}\lambda _{\mu v}\left( x \right) \left[ \partial ^{\mu}\phi \left( x \right) -\epsilon ^{\mu \sigma}\partial _{\sigma}\phi \left( x \right) \right] \left[ \partial ^v\phi \left( x \right) -\epsilon ^{v\rho}\partial _{\rho}\phi \left( x \right) \right] \right\}
$$
先分别计算其中的积分
$$
    S_1=\int{d}^2x\left[ \partial _{\mu}\phi \left( x \right) \partial ^{\mu}\phi \left( x \right) \right]
$$
$$
    S_2=\int{d}^2x\left\{ \lambda _{\mu v}\left( x \right) \left[ \partial ^{\mu}\phi \left( x \right) -\epsilon ^{\mu \sigma}\partial _{\sigma}\phi \left( x \right) \right] \left[ \partial ^v\phi \left( x \right) -\epsilon ^{v\rho}\partial _{\rho}\phi \left( x \right) \right] \right\}
$$
$$
    \because \phi \left( x \right) =\mathcal{F}\left( \phi \left( k \right) \right) ,\lambda _{\mu v}\left( x \right) =\mathcal{F}\left( \lambda _{\mu v}\left( k \right) \right)
$$
换元,令
$$
    g_{\mu}=-ik_{\mu}\phi \left( k \right) ,g^{\mu}=-ik^{\mu}\phi \left( k \right)
$$
利用导数定理\ref{11dao}
$$
    \because \mathcal{F}\left( -ik_{\mu}\phi \left( k \right) \right) =\partial _{\mu}\phi \left( x \right) ,\mathcal{F}\left( -ik^{\mu}\phi \left( k \right) \right) =\partial ^{\mu}\phi \left( x \right)
$$
利用卷积定理\ref{11juan}
$$
    S_1=\int{d}^2x\mathcal{F}\left( g_{\mu} \right) \mathcal{F}\left( g^{\mu} \right) =\frac{1}{2\pi}\int{d}^2x\mathcal{F}\left( g_{\mu}*g^{\mu} \right)
$$
最后利用\ref{11delta}
$$
    \therefore S_1=\left. \left( g_{\mu}*g^{\mu} \right) \right|_{k=0}=\int{d}^2\xi g_{\mu}\left( \xi \right) g^{\mu}\left( 0-\xi \right) =\int{d}^2k\left( -ik_{\mu} \right) \phi \left( k \right) \left( -ik^{\mu} \right) \phi \left( k \right)
$$
同理我们设
$$
    \eta ^{\mu}=ik^{\mu}\phi \left( k \right) -\epsilon ^{\mu \sigma}ik_{\sigma}\phi \left( k \right) ,\eta ^{\nu}=ik^{\nu}\phi \left( k \right) -\epsilon ^{v\rho}ik_{\rho}\phi \left( k \right)
$$
有
$$
    S_2=\int{d}^2x\mathcal{F}\left( \lambda _{\mu v}\left( k \right) \right) \mathcal{F}\left( \eta ^{\mu} \right) \mathcal{F}\left( \eta ^{\nu} \right) =\frac{1}{4\pi ^2}\int{d}^2x\mathcal{F}\left( \lambda _{\mu v}\left( k \right) *\eta ^{\mu}*\eta ^{\nu} \right)
$$
$$
    S_2=\frac{1}{2\pi}\left. \lambda _{\mu v}\left( k \right) *\eta ^{\mu}*\eta ^{\nu} \right|_{k=0}=\frac{1}{2\pi}\int{d}^2\xi \int{d}^2\zeta \lambda _{\mu v}\left( \zeta +\xi \right) \eta ^{\mu}\left( 0-\xi \right) \eta ^{\nu}\left( 0-\zeta \right)
$$
即为
$$
    S_2=\frac{1}{2\pi}\int{d}^2k\int{d}^2k'\lambda _{\mu v}\left( -k-k' \right) \left[ ik^{\mu}\phi \left( k \right) -\epsilon ^{\mu \sigma}ik_{\sigma}\phi \left( k \right) \right] \left[ ik^{\nu}\phi \left( k' \right) -\epsilon ^{v\rho}ik_{\rho}\phi \left( k' \right) \right]
$$

现在我们终于可以代回到原表达式
$$
    \begin{aligned}
        S_{m}
        = & -\frac{1}{2} \int d^{2} k\left(-i k_{\mu}\right)\left(i k^{\mu}\right) \phi(k) \phi(-k)                                                                                                                                                                      \\
          & +\frac{1}{4 \pi} \int d^{2} k d^{2} k^{\prime} \lambda_{\mu \nu}\left(-k-k^{\prime}\right)\left(i k^{\mu}-\epsilon^{\mu \sigma} i k_{\sigma}\right)\left(i k^{\prime \nu}-\epsilon^{\nu \rho} i k_{\rho}^{\prime}\right) \phi(k) \phi\left(k^{\prime}\right)
    \end{aligned}
$$

这就是结果

\section{第十二题}
\subsection{}
$$
    g_1(t)=\left\{\begin{array}{cc}
        2 t / T    & 0 \leq t<T / 2      \\
        2(1-t / T) & T / 2 \leq t \leq T
    \end{array}\right.
$$
根据周期函数
$$
    \mathcal{L}\left( g\left( t \right) \right) =
    \frac{1}{1-e^{-p T}} \int_{0}^{T} e^{-p t} g(t) d t
$$
计算得
$$
    \int_0^{T/2}{\exp \left( -pt \right) \frac{2t}{T}dt=\frac{2}{T}}\left[ \frac{1}{p^2}-e^{-Tp/2}\left( \frac{T}{2p}+\frac{1}{p^2} \right) \right]
$$
$$
    -\int_{T/2}^T{\exp \left( -pt \right) \frac{2t}{T}dt=\frac{2}{T}}e^{-Tp/2}\left[ \left( 1-e^{-Tp/2} \right) \left( \frac{T}{2p}+\frac{1}{p^2} \right) \right]
$$
$$
    \int_{T/2}^T{2\exp \left( -pt \right) dt=\frac{2}{p}}\left( e^{-Tp/2}-e^{-Tp} \right)
$$
代入化简得
$$
    \mathcal{L}\left( g_1\left( t \right) \right) =\frac{2}{Tp^2}\tan\text{h}\left( \frac{Tp}{4} \right)
$$
\subsection{}
$$
    g_2\left( t \right) =\frac{2}{T}\left[ \text{tH}\left( \text{t} \right) +2\sum_{n=1}^{\infty}{\left( -1 \right)}^n\left( t-\frac{1}{2}nT \right) H\left( t-\frac{1}{2}nT \right) \right]
$$
$$
    \mathcal{L}\left( g_2\left( t \right) \right) =\frac{2}{Tp^2}\left[ 1+2\sum_{n=1}^{\infty}{\left( -1 \right)}^n\exp \left( -\frac{1}{2}nT \right) \right]
$$
如果$g_1(t)=g_2(t)$,比照系数令$x=Tp/4$换元有
$$
    \tanh x=1+2 \sum_{n=0}^{\infty}(-1)^{n} e^{-2 n x}
$$
这正是我们想证明的。因此只需证明$g_1(t)=g_2(t)$.

下面证明$g_1(t)=g_2(t)$

我们首先考察区间$[0,T/2)$,对于$t\in [0,T/2]$由于全部的$H(t-1/2nT)=0$因此二者显然相等。

然后考察区间$[T/2,T)$,对于$t\in [0,T/2]$求和只保留到$n=1$,有
$$
    g_2=\frac{2}{T}\left[ t-2\left( t-\frac{1}{2}T \right) \right] =2-\frac{2t}{T}=g_1
$$

综上二者在$t\in [0,T)$上相等。

一般地,对于$n>0$情形,仅当$t\geq 1/2nT$时,$H(t-1/2nT)=1$,其余情况为0。如果我们只考察$t$非负半轴情况,求和可以写成:
$$
    g_2\left( t \right) =\frac{2}{T}\left[ t +2\sum_{n=1}^{\left[ \frac{2t}{T} \right]}{\left( -1 \right)}^n\left( t-\frac{1}{2}nT \right) \right]
$$
其中$\left[ \frac{2t}{T} \right]$表示括号内值向下取整。

接下来证明$g_2(t)$也以$T$为周期,既$g_2(t)=g_2(t+T)$
$$
    \begin{aligned}
        g_2\left( t+T \right)
         & =\frac{2}{T}\left[ \text{t}+T+2\sum_{n=1}^{\left[ \frac{2t}{T}+2 \right]}{\left( -1 \right)}^n\left( t-\frac{1}{2}nT+T \right) \right]                         \\
         & =\frac{2}{T}\left[ \text{t}+T-\left( 2t+T \right) +2t+2\sum_{n=3}^{\left[ \frac{2t}{T}+2 \right]}{\left( -1 \right)}^n\left( t-\frac{1}{2}nT+T \right) \right] \\
         & =\frac{2}{T}\left[ t+2\sum_{n=1}^{\left[ \frac{2t}{T} \right]}{\left( -1 \right)}^n\left( t-\frac{1}{2}nT \right) \right]                                      \\
         & =g_2\left( t \right)
    \end{aligned}
$$

由于在$[0,T)$区间相等并都具有以$T$为周期的周期性,因此$g_1(t)=g_2(t)$,这正是我们需要的。
于是
$$
    \tanh x=1+2 \sum_{n=0}^{\infty}(-1)^{n} e^{-2 n x}
$$
得证


\section{第十三题}
以下的拉普拉斯方程的表达式都基于
$$
    \frac{1}{\sqrt{|g|}} \partial_{\mu}\left(g^{\mu v} \sqrt{|g|} \partial_{v} \psi\right)=0
$$
在正交曲线坐标系下为
$$
    \sum_{\mu =1}^3{\frac{1}{|h_1h_2h_3|}\partial _{\mu}\left( \frac{1}{h_{\mu}}\sqrt{|h_1h_2h_3|}\partial _{\mu}\psi \right)}=0
$$

\subsection{球坐标}
$$
    h_r=1,h_{\theta}=r,h_{\varphi}=r\sin \theta
$$
拉普拉斯方程为:
\[
    \frac{1}{r^{2}} \frac{\partial}{\partial r}\left(r^{2} \frac{\partial u}{\partial r}\right)+\frac{1}{r^{2} \sin \theta} \frac{\partial}{\partial \theta}\left(\sin \theta \frac{\partial u}{\partial \theta}\right)+\frac{1}{r^{2} \sin ^{2} \theta} \frac{\partial^{2} u}{\partial \varphi^{2}}=0
\]
首先,设
\[
    u(r, \theta, \varphi)=R(r) \mathrm{Y}(\theta, \varphi)
\]
\[
    \frac{1}{R} \frac{\mathrm{d}}{\mathrm{d} r}\left(r^{2} \frac{\mathrm{d} R}{\mathrm{d} r}\right)=-\frac{1}{\mathrm{Y} \sin \theta} \frac{\partial}{\partial \theta}\left(\sin \theta \frac{\partial \mathrm{Y}}{\partial \theta}\right)-\frac{1}{\mathrm{Y} \sin ^{2} \theta} \frac{\partial^{2} \mathrm{Y}}{\partial \varphi^{2}}=l(l+1)
\]
分解为两个方程
\[
    \frac{d}{d r}\left(r^{2} \frac{d R}{d r}\right)-l(l+1) R \equiv 0
\]
\[
    \frac{1}{\sin \theta} \frac{\partial}{\partial \theta}\left(\sin \theta \frac{\partial Y}{\partial \theta}\right)+\frac{1}{\sin ^{2} \theta} \frac{\partial^{2} Y}{\partial \varphi^{2}}+l(l+1) Y=0
\]
进一步设
\[
    \mathrm{Y}(\theta, \varphi)=\Theta(\theta) \Phi(\varphi)
\]
\[
    \frac{\sin \theta}{\Theta} \frac{\mathrm{d}}{\mathrm{d} \theta}\left(\sin \theta \frac{\mathrm{d} \Theta}{\mathrm{d} \theta}\right)+l(l+1) \sin ^{2} \theta=-\frac{1}{\Phi} \frac{\mathrm{d}^{2} \Phi}{\mathrm{d} \varphi^{2}}=\lambda
\]
分解为两个常微分方程:
\[
    \Phi^{\prime \prime}+\lambda \Phi=0
\]
\[
    \sin \theta \frac{\mathrm{d}}{\mathrm{d} \theta}\left(\sin \theta \frac{\mathrm{d} \Theta}{\mathrm{d} \theta}\right)+\left[l(l+1) \sin ^{2} \theta-\lambda\right] \theta=0
\]
\subsection{柱坐标}
$$
    h_{\rho}=1,h_{\varphi}=\rho ,h_z=1
$$

柱坐标的拉普拉斯方程表示为
\[
    \frac{1}{\rho} \frac{\partial}{\partial \rho}\left(\rho \frac{\partial u}{\partial \rho}\right)+\frac{1}{\rho^{2}} \frac{\partial^{2} u}{\partial \varphi^{2}}+\frac{\partial^{2} u}{\partial z^{2}}=0
\]
设
\[
    u(\rho, \varphi, z)=R(\rho) \Phi(\varphi) Z(z)
\]
代入得到
\[
    \Phi^{\prime \prime}+\lambda \Phi=0
\]
以及
\[
    \frac{\rho^{2}}{R} \frac{\mathrm{d}^{2} R}{\mathrm{d} \rho^{2}}+\frac{\rho}{R} \frac{\mathrm{d} R}{\mathrm{d} \rho}+\rho^{2} \frac{Z^{\prime \prime}}{Z}=\lambda
\]
分离有:
\[
    \frac{1}{R} \frac{\mathrm{d}^{2} R}{\mathrm{d} \rho^{2}}+\frac{1}{\rho R} \frac{\mathrm{d} R}{\mathrm{d} \rho}-\frac{m^{2}}{\rho^{2}}=-\frac{Z^{\prime \prime}}{Z}=-\mu
\]
两个常微分方程
\[
    Z^{\prime \prime}-\mu Z=0
\]
\[
    \frac{\mathrm{d}^{2} R}{\mathrm{d} \rho^{2}}+\frac{1}{\rho} \frac{\mathrm{d} R}{\mathrm{d} \rho}+\left(\mu-\frac{m^{2}}{\rho^{2}}\right) R=0
\]
\subsection{椭圆柱坐标}
标准表示为
$$
x=a \xi \eta, \quad y=a \sqrt{\left(\xi^{2}-1\right)\left(1-\eta^{2}\right)}, \quad z=z
$$
换元,设
$$\xi=\operatorname{ch} u, \quad \eta=\cos v$$
有
$$
    h_{u}^{2}=h_{v}^{2}=a^{2}\left(\operatorname{ch}^{2} u-\cos ^{2} v\right), \quad h_{z}=1
$$
拉普拉斯方程表示为
$$
    \nabla^{2} \Phi=\frac{1}{a^{2}\left(\operatorname{ch}^{2} u-\cos ^{2} v\right)}\left\{\frac{\partial^{2} \Phi}{\partial u^{2}}+\frac{\partial^{2} \Phi}{\partial v^{2}}\right\}+\frac{\partial^{2} \Phi}{\partial z^{2}}=0
$$
设
$$
\Phi \left( u,v,z \right) =U\left( u \right) V\left( v \right) Z\left( z \right) 
$$
代入有
$$
\frac{1}{a^2\left( \operatorname{ch}^2u-\cos ^2v \right)}\left\{ \frac{1}{U}\frac{d^2U}{du^2}+\frac{1}{V}\frac{d^2V}{dv^2} \right\} +\frac{1}{Z}\frac{d^2Z}{dz^2}=0
$$
分离出
$$
\frac{d^2Z}{dz^2}+\lambda Z=0
$$
另有
$$
\frac{1}{U}\frac{d^2U}{du^2}+\frac{1}{V}\frac{d^2V}{dv^2}-\lambda a^2\left( \operatorname{ch}^2u-\cos ^2v \right) =0
$$
分离出
$$
\frac{d^2U}{du^2}-\left( \lambda a^2ch^2u+\mu \right) U=0
$$
$$
\frac{d^2V}{dv^2}+\left( \lambda a^2\cos ^2v+\mu \right) V=0
$$

\subsection{椭球坐标}
\[
    x^{2}=\frac{\left(a^{2}+\lambda\right)\left(a^{2}+\mu\right)\left(a^{2}+v\right)}{\left(a^{2}-b^{2}\right)\left(a^{2}-c^{2}\right)}     
\]
\[
    y^{2}=\frac{\left(b^{2}+\lambda\right)\left(b^{2}+\mu\right)\left(b^{2}+v\right)}{\left(b^{2}-c^{2}\right)\left(b^{2}-a^{2}\right)}     
\]
\[
    z^{2}=\frac{\left(c^{2}+\lambda\right)\left(c^{2}+\mu\right)\left(c^{2}+v\right)}{\left(c^{2}-a^{2}\right)\left(c^{2}-b^{2}\right)}
\]
$\lambda, \mu, \nu$ 的变化范围为
\[
    \lambda>-c^{2}>\mu>-b^{2}>\nu>-a^{2}     
\]
若设
\[
    \varphi(\theta)=\left(a^{2}+\theta\right)\left(b^{2}+\theta\right)\left(c^{2}+\theta\right)
\]
有
\[
    h_{\lambda}^{2}=\frac{(\lambda-\mu)(\lambda-\nu)}{4 \varphi(\lambda)}
\]
\[
    h_{\mu}^{2}=\frac{(\mu-\lambda)(\mu-\nu)}{4 \varphi(\mu)}        
\]
\[
    h_{\nu}^{2}=\frac{(\nu-\lambda)(\nu-\mu)}{4 \varphi(\nu)}    
\]
\[
\begin{aligned}
    \nabla^{2} \Phi
    &=\frac{2}{(\lambda-\mu)(\lambda-\nu)(\mu-\nu)}\left[(\mu-\nu) \sqrt{\varphi(\lambda)} \frac{\partial}{\partial \lambda}\left(\sqrt{\varphi(\lambda)} \frac{\partial \Phi}{\partial \lambda}\right)\right. \\
    &+(\lambda-\nu) \sqrt{-\varphi(\mu)} \frac{\partial}{\partial \mu}\left(\sqrt{-\varphi(\mu)} \frac{\partial \Phi}{\partial \mu}\right) \\
    &+\left.(\lambda-\mu) \sqrt{\varphi(\nu)} \frac{\partial}{\partial \nu}\left(\sqrt{\varphi(\nu)} \frac{\partial \Phi}{\partial \nu}\right)\right]=0
\end{aligned}
\]
设$\Phi=\Lambda(\lambda) M(\mu) N(\nu)$代入有
\[
    \begin{aligned}
        \frac{\mu-\nu}{\Lambda} 
        & \sqrt{\varphi(\lambda)} \frac{\mathrm{d}}{\mathrm{d} \lambda}\left(\sqrt{\varphi(\lambda)} \frac{\mathrm{d} \Lambda}{\mathrm{d} \lambda}\right)+\frac{\lambda-\nu}{M} \sqrt{-\varphi(\mu)} \frac{\mathrm{d}}{\mathrm{d} \mu}\left(\sqrt{-\varphi(\mu)} \frac{\mathrm{d} M}{\mathrm{d} \mu}\right) \\
        &+\frac{\lambda-\mu}{N} \sqrt{\varphi(\nu)} \frac{\mathrm{d}}{\mathrm{d} \nu}\left(\sqrt{\varphi(\nu)} \frac{\mathrm{d} N}{\mathrm{d} \nu}\right)=0
        \end{aligned}    
\]
注意到有恒等式
\[
(\mu-\nu)(K \lambda+C)+(\nu-\lambda)(K \mu+C)+(\lambda-\mu)(K \nu+C) \equiv 0
\]
其中 K 和C 为常数,比较系数得
\[
4 \sqrt{\varphi(\lambda)} \frac{\mathrm{d}}{\mathrm{d} \lambda}\left(\sqrt{\varphi(\lambda)} \frac{\mathrm{d} \Lambda}{\mathrm{d} \lambda}\right)=(K \lambda+C) \Lambda
\]
设$K=n(n+1)$,分离变量的三个方程为:
$$
4\sqrt{\varphi \left( \lambda \right)}\frac{\text{d}}{\text{d}\lambda}\left( \sqrt{\varphi \left( \lambda \right)}\frac{\text{d}\Lambda}{\text{d}\lambda} \right) =\left[ \left( n\left( n+1 \right) \right) \lambda +C \right] \Lambda 
$$
$$
4\sqrt{\varphi \left( \mu \right)}\frac{\text{d}}{\text{d}\mu}\left( \sqrt{\varphi \left( \mu \right)}\frac{\text{d}M}{\text{d}\mu} \right) =\left[ \left( n\left( n+1 \right) \right) \mu +C \right] M
$$
$$
4\sqrt{\varphi \left( \nu \right)}\frac{\text{d}}{\text{d}\nu}\left( \sqrt{\varphi \left( \nu \right)}\frac{\text{d}N}{\text{d}\nu} \right) =\left[ \left( n\left( n+1 \right) \right) \nu +C \right] N
$$
\subsection{锥面坐标}
\subsection{抛物线柱坐标}
\section{第十四题}
\end{document}