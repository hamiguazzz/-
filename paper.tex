数学物理方法课程分成两个部分,一是复变函数论,二是数学物理方程。
\setlength{\parindent}{2em}

第一部分首先从复数的引入开始,拓展到复函数。与实分析类似,讨论复变函数的导数和积分,并针对复函数的性质引入柯西积分公式。​在复变函数领域,我们可以通过洛朗展开将泰勒展开进行推广,通过洛朗展开,我们不仅能奇点进行更好的分类,还可以更方便的计算包围奇点区域上的积分,即留数定理,这在某种意义上也是柯西公式的推广。我们利用留数定理不仅能解决复积分,也对实变的积分起了很大帮助。接下来我们将实变函的傅立叶级数推广到到复变函数领域,并从离散的求和变成了连续的积分,得到了傅立叶变换。为了解决奇点上积分无穷大问题和$\exp(ikx)$的变换,引入了广义函数——$\delta$函数,为了解决一部分函数不能变换的问题,我们引入拉普拉斯变换,并分别研究了它们的性质和应用,许多问题因此而变得简单。

第二部分从介绍一些物理情形引入偏微分方程开始,并引入了初始条件和边界条件给出了定解条件,介绍了分类,并针对简单的(半)无界可分离算子情况给出了达朗贝尔公式。此后的讨论主要建立在分离变量的基础上,研究了许多不同方程分离变量的结果。介绍的解PDE一般步骤为利用分离变量法,将偏微分方程转化为带有未定常数和约束条件的常微分方程,即本征值问题,由于讨论到的本征值问题都是施图姆刘维尔型,解具有正交完备性,可以做广义复立叶展开适配边界条件和初始条件,得到解的一般形式。

基于在第一部分讨论的级数和奇点知识,利用常点邻域上的级数解法和正则奇点邻域上的级数解法,求出了勒让德函数和贝塞尔函数,在求解分离变量后得到的本征值方程中有重要应用。

接下来与特殊函数课类似详细研究了勒让德函数和贝塞尔函数(球柱函数)的性质:表示,模长,渐进行为,正交性,广义傅里叶展开(其实是S-L问题的性质),对于贝塞尔函数还有其零点,以及二者一些变形表示。

一句话简单总结为列PDE与定解条件+分离变量+本征值方程+正交完备特殊函数+广义傅里叶展开



\setlength{\parindent}{0em}


​
